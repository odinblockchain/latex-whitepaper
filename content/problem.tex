\section{​The Problem}
\epigraph{\itshape
``At every door-way,\\
ere one enters, \\
one should spy ‘round,\\
one should pry ‘round,\\
for uncertain is the witting\\
that there be no foeman sitting,\\
within, before one on the floor.''}{intro, H\'av\'amal}

ODIN understood the importance of Wisdom. The V\"olusp\`a describes him in the act of  sacrificing one of his eyes in exchange for a single drink from Mimir's well of Wisdom, and thus becoming wiser himself as well.

\subsection{Privacy}
In today's digital age, we find parallels to this. ODIN Blockchain intends to shield the proponents of liberty of thought and ideas from the vision of these aforementioned ``foemen'' and their prying eyes. As Odin sacrificed part of his vision in exchange for knowledge, we plan on obscuring from vision your private conversations and exchanges of thoughts and knowledge. 

Governmental and private institutions are becoming increasingly more adept at gathering data on citizens. It has, in fact, become both a \href{https://patents.google.com/patent/US20110087529A1/en}{science} and a \href{https://patents.google.com/patent/US5974396A/en}{business}. What data to gather and analyze it for practical use. \href{https://www.theguardian.com/news/series/cambridge-analytica-files}{The contemporary media} is full of articles on how \href{https://wikileaks.org/}{lines are being crossed} and how grey areas lack definition.  In addition, passed legislature is allowing for such wide digital surveillance on citizens without justifiable cause, that the old adage ``innocent until proven guilty'' no longer even applies. These laws appear to assume malicious intent from each and every citizen, allowing surveillance on a level that presupposes that we are already suspects in ongoing investigations.

By removing the threat of persecution or worse from the equation, we are encouraging people around the world to contribute to our collective development without fear of censorship.

We believe that, in order to enable a healthy exchange of ideas and thoughts, freedom from persecution and sanctions must be guaranteed when people exercises their simple right to speak. 

Unfortunately, not all institutions around the globe; governmental, private or otherwise, agree with this. Some wish to suppress the exchange of ideas, as it could pose a threat to the powerful. We consider this to be unhealthy for human development as a whole; not just politically, but technologically, spiritually and philosophically as well.

Globally, there is an \href{https://foreignpolicy.com/2010/05/07/the-worlds-top-dissidents/}{abundance of dissidents} that are not allowed to speak under threat of severe consequences to their lives and possibly more.

The ODIN Blockchain will operate in line with Article 17 of the \href{https://www.ohchr.org/EN/ProfessionalInterest/Pages/CCPR.aspx}{International Covenant on Civil and Political Rights} of the United Nations of 1966, stating that: ``No one shall be subjected to arbitrary or unlawful interference with his privacy, family, home or correspondence, nor to unlawful attacks on his honour and reputation. Everyone has the right to the protection of the law against such interference or attacks.''

In addition, large corporations today have a massive points of vulnerability: their centralized control and storage systems delivered by server farms. All data is routed through them, stored for years due to the widely imposed \href{http://fra.europa.eu/en/theme/information-society-privacy-and-data-protection/data-retention}{Data} \href{https://sydney.edu.au/news-opinion/news/2017/07/31/new-data-retention-law-seriously-invades-our-privacy.html}{Retention} Laws, and consequently may be subjected to searches from governmental agencies or even mined by the corporations themselves.

This approach is also increasingly compromised with malicious intent. One can list a litany of such hacks, including at least \href{http://money.cnn.com/2016/09/22/technology/yahoo-data-breach/?iid=EL}{500 million accounts} that had been stolen from Yahoo or \href{http://money.cnn.com/2017/10/02/technology/business/equifax-million-more-impacted/index.html?iid=EL}{145.5 million people} that were compromised in a breach through Equifax.

With ODIN applications, this would be an impossibility. We have a decentralized network of servers where all data that is routed through masternodes is anonymized.

\subsection{The Internet is Progressively Going Mobile}
By and large, the greatest threat to one's personal online data and digital fingerprint is because of the development of technologies which allow ease of use. This same ease of use makes it very easy for one to forget exactly how much information is stored and accessed on their mobile devices.

Globally, \href{https://www.ericsson.com/en/mobility-report/reports/june-2018/mobile-subscriptions-worldwide-q1-2018}{7.9 billion mobile phone subscriptions} were active during Q1 of 2018. Mobile internet traffic \href{https://www.ericsson.com/en/mobility-report/reports/june-2018/mobile-traffic-report-q1-2018}{increased by 54\%} between Q1 2017 and Q1 2018.

Mobile devices have become an integral part of the average person's daily routine. From reading the news on a daily commute to performing online payments, to contacting others through calls and text. 

The great ease and comfort a mobile device provides lulls one into a false sense of security where it is easy to fall victim to a range of issues including but not limited to:
\begin{itemize}
   \item Website URLs that are not displayed in full by default, increasing the risk of a user entering a malicious site
   \item Popup windows that start to run in the background
   \item Applications that ask for more permissions than necessary without expanding on why they require it
\end{itemize}
In addition, the underlying operating systems of mobile devices are less understood by users compared to PCs and laptops, especially with regards to how much private and personal data is actually held or accessible through a smartphone.
 
In conclusion, the problems we have defined are:
\begin{itemize}
   \item Privacy infringement by governmental agencies
   \item Privacy infringement by private institutions as the recent \href{https://www.bbc.com/news/technology-43465968}{Cambridge Analytica} scandal which attempted to influence voters through data mining personal information
   \item Censorship, as opposed to free speech, even going so far as physically abusing and threatening families of dissidents
   \item A distinct lack of private cyber security measures specifically when it comes to mobile devices
   \item The centralization of data, which increases the risk of hacks compromising millions of individuals
   \item A single compromised device can bypass all security measures one has taken across multiple devices.
\end{itemize}
As privacy, anonymity and cyber security are the three key points necessary to ensure a user's device safety, we have opted to encapsulate all three of these points under one ODIN ecosystem. 
