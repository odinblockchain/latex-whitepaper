\section{The Solution}\hypertarget{solution}{}

People have a decent understanding of how to secure their PC and/or laptop. Most people will run an antivirus program, have their firewalls active, and exercise common sense to know what is safe to download and what is not. 

However, as stated previously, the majority of internet access today is no longer from desktops or laptops, but rather by mobile devices.  In 2015, \href{https://adwords.googleblog.com/2015/05/building-for-next-moment.html}{Google reported} more than 50\% of all searches were done on a mobile device in 10 countries, including the US and Japan. 

This exceeded the amount of desktop searches for the first time in history.

According to StatCounter, starting in November 2016 general mobile internet usage \href{http://gs.statcounter.com/#mobile_vs_desktop-ww-monthly-201611-201611-bar}{surpassed 50\% of web traffic}.

\begin{figure}[!h]
\centering
\begin{tikzpicture}
   \begin{axis}[
      ybar,
      bar width = 30,
      width=12.6cm,
      height=10cm,      
%       xlabel={year},
      ylabel={Number of mobile phone users, in billions},      
      symbolic x coords = {2015,2016,2017,2018,2019,2020+},
      ymin=0,
      ymax=6,
      nodes near coords,
      ]
      \addplot [draw=blue,fill=blue!50]
      coordinates {(2015,4.15) (2016,4.3) (2017,4.43) (2018,4.57) (2019,4.68) (2020+,4.78)};
   \end{axis}
\end{tikzpicture}
\caption*{Number of mobile phone users worldwide from 2015 to 2020 (in billions, \href{https://www.statista.com}{Statista.com})}
\end{figure}

Coupled with the \href{http://gs.statcounter.com/platform-market-share/desktop-mobile-tablet}{low entry threshold} a mobile device provides increased use of mobile devices in developing countries (more people have mobile phones \href{https://www.economist.com/graphic-detail/2017/11/08/in-much-of-sub-saharan-africa-mobile-phones-are-more-common-than-access-to-electricity}{than access to electricity} in Africa) it is becoming clear that mobile platforms will increase their market share in the future, to the point of becoming the absolute preferred method.

Taking the increasing popularity of mobile devices into account, it therefore makes sense for ODIN to focus its efforts on securing and anonymizing applications, data and services that primarily utilize the mobile platform.

ODIN will support applications that allow everything that could be done before, but without compromising your privacy or online security. It will also facilitate businesses and social models we have yet to dream about---similar to Sir Timothy John Berners-Lee being unable to anticipate game changing services like Netflix or Uber when he first \href{https://en.wikipedia.org/wiki/Tim_Berners-Lee}{invented the World Wide Web}.

Our blockchain will allow for the decentralization of applications and services avoiding the pitfalls of running applications through central organizations and server farms. Instead, data will be routed through any masternodes ran by the holders of the ODIN coin.

A range of toolkits will be created to lead you through a series of steps to build, test and launch your Dapps, including but not limited to:
\begin{itemize}
   \item Specific developer tools focusing on native languages
   \item UX designing and testing
   \item Prototyping and MVPs
   \item Business concept validation 
   \item Usability testing
   \item Marketing and promotion
   \item Agile implementation strategies
   \item OKR frameworks for aligned growth
   \item Acceptance testing/coding standards/documentation standards
   \item Mobile applications (on- or off-chain applications)
   \item How to support an online community
   \item Local exchange trading systems
   \item Community coins
   \item Growing a healthy community
\end{itemize}

A support infrastructure including but not limited to:
\begin{itemize}
   \item An active peer support developer community
   \item Financial assistance through community voted projects available through the community Vanir (\href{https://docs.google.com/document/d/15YAuXnc-3y06keG302AfK-6QCZ7PkonSIGuMRxncdcw/edit?disco=AAAACD6XbTI&ts=5b47a26b#heading=h.c0p3cw59hwku}{community portal}) supported via the OPL
   \item Education and partnership development
   \begin{itemize}
      \item Active links to universities and incubators
      \item Hosting hackathons
      \item Attending meetups to engage other projects and/or developers that may be interested in developing on the ODIN Blockchain
   \end{itemize}
\end{itemize}

We intend to bring the possibilities of the ODIN Blockchain to a greater audience, expanding far beyond the reach of crypto insiders. We will do this through focusing on the following three key areas: \textit{innovation}, \textit{intuitiveness}, and \textit{integrity}. 

\subsection{Innovation}
There are new and emerging opportunities for organizations and developers in all sectors to create and deliver compelling products, projects and services for their customers using the power of disruptive innovation brought by the use of blockchain technologies and by \textit{nurturing collaboration}.

The ODIN Alliance will advance ideas through research and peer-to-peer blockchain development with a rich and supportive range of partners, developers, business experts, and philanthropists. We aim to bridge the divide between Entrepreneurs (for profit and socially conscious) and innovative, agile, visionary developers.  

We empower all contributors to collaborate with us in shaping the future of ODIN.

\subsection{​Intuitive}
We understand very well the implications of using blockchain technology. As familiar as it is, it remains a new area of technology and thus people are generally reserved or even skeptical about its use case.

We intend to overturn this thinking by helping create applications that are user friendly, and that support and develop third-parties in creating out-of-the-box, `plug-and-play' software.

We aim to do this with our focus on usability to \textit{make blockchain usable}.

We will embody human-centered design across all our projects and frameworks in order to ensure intuitive ease of use and simplicity of experience. Supportive toolkits as expressed above are being built across all areas, from development through to design and community collaboration. These showcase ODIN's capabilities and help spur the growth of ODIN DApps and a library of smart contracts.

\subsubsection{Why is usability important?}
Usability is the measure of the quality of a user's experience when interacting with a product or system---whether a web site, software application, mobile technology, or any user-operated device.

There are many definitions for usability, but four elements can be broadly considered and which ODIN aims to address:
\begin{itemize}
   \item Intuitive and easy to learn
   \item Efficient to use
   \item Errors can be recovered from quickly
   \item Easy to remember
\end{itemize}
    
Whilst the end product Dapps will be out of our control, or more correctly in the control of the community, by focusing on usability we are more able to support creating projects, products and services where users will be satisfied, enjoy their interaction and achieve goals effectively and efficiently.  This will lead to more confidence and trust in what we are accomplishing.

Satisfied users are loyal users and increase the likelihood of recommending your product or service to others.

\subsubsection{The benefits of usability}
By focusing on usability, you will be benefited in many ways including:
\begin{itemize}
   \item Reduced development time and costs
   \item Reduced support costs
   \item Reduced user errors
   \item Reduced training time and costs
   \item Return on Investment
\end{itemize}

\subsubsection{​Integrity}

Whilst we have placed our focus on privacy and anonymization, the governance structure itself will be open and transparent. We will not ask people to blindly trust us to guide them, and consequently we will be open about our own goals and manners of achieving them. We always aim to `do the right things.'

This includes helping establish well respected open-source repositories and shaping governance and consensus decision-making models, so that we evolve into a truly decentralized community led organization.

Amongst others, we will:
\begin{itemize}
   \item Be transparent about funding and expenditures
   \item Create a community ethics board
   \item Welcome community suggestions
   \item Have the community to vote on proposals, effectively making OPL a catalyst for community initiatives and proponents to ensure all goes according to agreement, and making sure all funding is allocated to those projects the community chooses.
\end{itemize}
   
\subsection{​The ODIN Community}
As stated in the preface, the heart of ODIN is its community. ODIN will focus on four main communities:
\begin{itemize}
   \item The infrastructure community
   \item The Dapps community
   \item The end user community 
   \item The entrepreneurial community
\end{itemize}
All of whom require their own information, and have different motivations and needs.

ODIN already has an active and vibrant community which we aim to extend. We have a dedicated community manager aided by a group of very committed community moderators.

We constantly strive to help facilitate a positive, friendly, non-judgmental community who is willing and eager to support each other. 

Ultimately, ODIN Blockchain is directed by the community. As part of our commitment to \href{https://odinblockchain.org/#integrity}{Integrity}, we always want to be engaged, transparent and responsive to the needs of the community. We want you to shape our ideas and designs, and will constantly provide ways to gather feedback and to listen and understand what the ODIN Blockchain community wants.

\subsection{ODIN Community Portal}
To be a truly community led project, ODIN will provide a decentralized, recurring method to allocate funds for the development of valuable ideas and to help visualize the future direction of the ODIN ecosystem.

Community members that own the required amount of Odin for a masternode will be able to exercise their right to vote on what projects they wish to see pursued. A level of active education and engagement is required to make informed decisions for the future of the ODIN Ecosystem. 

Every masternode is entitled to one vote. Therefore, if you hold two masternodes you are given two votes. This allows for all levels of investment and genuine interest in the project’s success to have a bigger say, and reduces the odds of success for malicious actors to intentionally vote on terrible projects, obligating the Foundation to provide them with funds.

The funds for the pursuing of these projects will be taken out of stake rewards, and transferred to an openly visible address held by the Foundation.

Before each proposal cycle, any community member or aspiring developer may submit proposals that deliver value to the ODIN ecosystem. Listing a proposal will require submission of a 25 ODIN fee, which will be burned.

One proposal cycle will last for one month.

All these proposals will be publically viewable, and it is then left up to the community to debate and investigate these proposals for themselves. Masternode owners may then vote on these proposals.

At the end of each proposal cycle, voting is closed and the budget is finalized before being distributed. At this point the Foundation will submit a 25 ODIN fee which is burned, thus finalizing the budget for the cycle.

Masternodes automatically rank the proposals based on net yes versus no \%. Only the top three projects are then funded. The projects receive funds in a proportional way from the money raised during that cycle to, at most, the total value of the particular project and the remaining funds (if any) in the community wallet are rolled over to the following month.

For example, if four projects were submitted receiving the following votes
\begin{itemize}
   \item Project one -- 50\% of the vote
   \item Project two -- 20\% of the vote
   \item Project three -- 27\% of the vote
   \item Project four -- 3\% of the vote
\end{itemize}
Then the funds would be allocated as
\begin{itemize}
   \item Project one -- 51\% (50\% + 1\% of project four)
   \item Project two -- 21\% (20\% + 1\% of project four)
   \item Project three -- 28\% (27\% + 1\% of project four)
\end{itemize}

If however any \% vote caused a project to receive a larger fund than is needed, that money is rolled over to be available to the following month's allocation.

For example
\begin{itemize}
   \item Project one is to receive 51\% of the funds (which was, for example, 5000 ODIN) but that project only requested 4000 ODIN, 1000 ODIN would remain in the wallet until the following month.
   \item Periodically if the community wallet grows to an `excess' size we will engage with the community as to what you would like to do with these funds (options could include, roll over, burn etc)
\end{itemize}
